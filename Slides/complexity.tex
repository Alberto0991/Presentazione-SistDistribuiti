\subsection{Complessità}

\begin{frame} {Numero di Messaggi}

\begin{itemize}
\setlength\itemsep{1.5em}

\item Sia $n_i$ il numero massimo di candidati che si trovano nello stage$=i$;
\item Ogni candidato ha un territorio di dimensione $i$ e tutti i territori sono disgiunti, quindi non ci possono essere più di $n_i\leq n/i$ candidati;
\item Ogni candidato potrà inviare una richiesta di sfida che potrà costare al massimo 4 messaggi.

\vskip2em
$\Rightarrow$ il massimo numero di messaggi generati dai candidati con stage $i$ è $4n/i$.

\end{itemize}

\end{frame}

\begin{frame} {Numero di Step}

Quanti passi servono per la terminazione?
\vskip2em
\begin{itemize}
\item Se un candidato conquista un territorio di dimensione $\lfloor\frac{n}{2}\rfloor+1$ allora nessun altro candidato può diventare \textit{leader};
\item Un candidato che raggiunge stage$=\lfloor\frac{n}{2}\rfloor+1$ diventa quindi \textit{leader} e potrà inviare un messaggio di terminazione in broadcast a tutti gli altri nodi.
\end{itemize}
\end{frame}

\begin{frame} {Costo complessivo(Messaggi-Tempo)}

Il numero totale di messaggi includendo la notifica è:
\vskip0.5em
$n-1+\sum_{i=1}^{n/2} 4n_i \leq n-1+4n\sum_{i=1}^{n/2} \frac{1}{i} = 4nH_{n/2}+n-1 $

\vskip1.5em
Ricordando che $log(1+n)\leq H_n \leq 1+logn$ si ha che:

\vskip0.5em
$M[CompleteElect] \leq 4n(1+log\frac{n}{2})+n-1 = 4n+4nlogn-4nlog2+n-1 = 4nlogn+2.22n-1 = 2.77log_2n +2.22n -1 $ 

\vskip1.5em
Non è difficile vedere che il tempo ideale del protocollo nel caso peggiore è lineare, ovvero:
\vskip0.5em
$T[CompleteElect]=O(n)$
\end{frame}

\begin{frame}{Fusione di strategie}

Combinando in maniera appropriata la strategia appena descritta con quella Na{\"i}ve si può costruire un protocollo che migliora la complessità temporale a scapito del numero di messaggi.
\vskip2em
% \begin{itemize} \setlength\itemsep{1.5em}
% \item costo ottimale in termini di messaggi : $O(nlogn)$
% \item Tempo : $O(n/logn)$ 
% \end{itemize}
 
\end{frame}

\begin{frame}{Compromesso Tempo - \#Messaggi}

È possibile implementare un protocollo che, $\forall k$ con $logn\leq k \leq n$, risolve il problema dell'elezione utilizzando $O(nk)$ messaggi e un tempo dell'ordine di $O(n/k)$ nel caso peggiore.\vskip2em
\begin{itemize}
 \setlength\itemsep{2em}
\item $k=logn \Rightarrow$ strategia Combinata (numero messaggi ottimali): 
\begin{itemize}
\item numero di messaggi : $O(nlogn)$
\item tempo : $O(n/logn)$
\end{itemize} 

\item $k=n \Rightarrow$ strategia Na{\"i}ve (tempo ottimale): 
\begin{itemize}
\item numero di messaggi : $O(n^2)$
\item tempo : $O(1)$
\end{itemize} 

\end{itemize} 

\end{frame}
