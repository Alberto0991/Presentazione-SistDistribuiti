

\section{Introduzione}
\begin{frame}{Una soluzione Na{\"i}ve}


\begin{itemize}
\setlength\itemsep{2em}

 \item Ogni nodo può autonomamente sapere chi è il leader semplicemente ricevendo l'id di tutti i suoi vicini.
 \item Numero di messaggi: O$(n^2$)
 \item Tempo: $O(1)$
\end{itemize}


\end{frame}

\section{Complete Elect}
\begin{frame}{Stati}


\begin{itemize}
\setlength\itemsep{2em}

 \item S = \{\textit{ASLEEP}, \textit{CANDIDATE},\textit{PASSIVE}, \textit{CAPTURED}, \textit{FOLLOWER}, \textit{LEADER}\};
%  \vskip2em
 \item \textit{S\ped{INIT}} = \{\textit{ASLEEP}\}; \textit{S\ped{TERM}} = \{\textit{FOLLOWER}, \textit{LEADER}\}.
%  \vskip2em
 \item Restrictions: IR $\cup$ CompleteGraph.
\end{itemize}
\end{frame}

\begin{frame}{Stato interno}

 Lo stato interno di un nodo è descritto da:
 
 \vskip2em
 
 \begin{itemize}
  \item id
  \item stage: il numero dello stato elettorale(numero di nodi catturati)
  \item owner: il nodo dal quale si è stati catturati
  \item owner\_stage: la stage del nodo dal quale si è stati catturati
 \end{itemize}
 

\end{frame}

\begin{frame}{Protocollo}
 \begin{itemize}
 \setlength\itemsep{2em}
  \item Inizialmente tutti i nodi sono \textit{ASLEEP}; spontaneamente un nodo \textit{ASLEEP} diventa \textit{CANDIDATE}.
  \item Un nodo \textit{CANDIDATE} prova a catturare, uno dopo l'altro tutti i suoi vicini. Se il nodo riesce a catturare un suo vicino aumenta il suo stage altrimenti diventa \textit{PASSIVE}
  e termina la fase di espansione.
  \item Quando un nodo \textit{CANDIDATE} raggiunge lo stage $\lfloor\frac{n}{2}\rfloor+1$ diventa \textit{LEADER} e lo comunica a tutti i suoi vicini che diventano \textit{FOLLOWER}.
 \end{itemize}
\end{frame}




\begin{frame}{Cattura}
Il nodo x manda una messaggio di cattura ad y
\begin{center}
\begin{tikzpicture}[->,>=stealth',shorten >=1pt,auto,node distance=2cm,
   thin,
   test/.style={rectangle,fill=white!20,draw,font=\sffamily\tiny\normalfont},
   leaf/.style={rectangle,fill=white!20,draw,font=\sffamily\tiny\normalfont},
   every edge/.style={font=\sffamily\tiny\normalfont},
    level 1/.style={sibling distance=2cm},
    level 2/.style={sibling distance=2.4cm},
    level 3/.style={sibling distance=3cm},
 ]
 
\tiny{

  \node[test] (1) {In che stato è y?}
    
   
    child{node[test]{$stage(x)<stage(y)?$} 
	  child{node[leaf]{Success} edge from parent node[right]{Yes} }
	  child{node[test]{$stage(x)=stage(y)\wedge id(x)<id(y)?$}  
	    child{node[leaf]{Success} edge from parent node[right]{Yes} }
	    child{node[leaf]{Failure} edge from parent node[right]{No}} edge from parent node[right]{No}}
	    edge from parent node[left]{Candidate}
 	    }
    child{ node[leaf]{Success} edge from parent node[left]{Asleep} }
    child{node[leaf]{Success} edge from parent node[left]{Passive} }
    child{node[test]{$stage(y)<stage(x)?$} 
	child{ node[leaf]{Failure} edge from parent node[right]{No} }
	child{node[test]{$stage(owner(y))<stage(x)?$} 
	      child{node[test]{$stage(owner(y))=stage(x) \wedge id(owner(y))>id(x)?$} 
		child{node[leaf]{Success} edge from parent node[right]{Yes}}
		child{node[leaf]{Failure} edge from parent node[right]{No}}	    
		edge from parent node[right]{No}}
 	    child{node[leaf]{Success} edge from parent node[left]{Yes}}
	  edge from parent node[right]{Yes}}
          edge from parent node[right]{Captured}
	  }

    ;
}

\end{tikzpicture}
\end{center} 
\end{frame}














 
