\tikzstyle{nodeStyle}=[->,>=stealth',shorten >=1pt,auto,node distance=1.5cm,
  thick,
  asleep_node/.style={circle,fill=white!20,draw,font=\sffamily\Large\bfseries},
  candidate_node/.style={circle,fill=green!20,draw,font=\sffamily\Large\bfseries},
  passive_node/.style={circle,fill=gray!20,draw,font=\sffamily\Large\bfseries},
  captured_node/.style={circle,fill=blue!20,draw,font=\sffamily\Large\bfseries},
  follower_node/.style={circle,fill=yellow!20,draw,font=\sffamily\Large\bfseries},
  leader_node/.style={circle,fill=red!50,draw,font=\sffamily\Large\bfseries},]

  \subsection{Il problema della sfida indiretta}
\begin{frame}{Il problema della sfida indiretta}
Supponiamo di essere in questa configurazione


\begin{center}
\begin{tikzpicture}[nodeStyle]

  \node[captured_node] (1)  {$43_{5}^2$};
  \node[candidate_node] (2)  [above right of=1] {$5^2$};
  \node[captured_node] (3)  [below right of=2] {$71_{6}^2$};
  \node[candidate_node] (4)  [above right of=3] {$6^2$};
  
  \node[candidate_node] (5)  [below of=1] {$4^2$};
  \node[candidate_node] (6)  [right of=5] {$3^2$};
  \node[captured_node] (7)  [below left of=5] {$13_{4}^2$};
  \node[captured_node] (8)  [right of=7] {$15_{3}^2$};

  

\end{tikzpicture}
\end{center} 
\end{frame}

\begin{frame}{Il problema della sfida indiretta}
 4 sfida 43 e prima che 5 abbia risposto arriva anche la richiesta di sfida da parte di 3

\begin{center}
\begin{tikzpicture}[nodeStyle]

  \node[captured_node] (1)  {$43_{3}^3$};
  \node[passive_node] (2)  [above right of=1] {$5^2$};
  \node[captured_node] (3)  [below right of=2] {$71_{6}^2$};
  \node[candidate_node] (4)  [above right of=3] {$6^2$};
  
  \node[candidate_node] (5)  [below of=1] {$4^3$};
  \node[candidate_node] (6)  [right of=5] {$3^3$};
  \node[captured_node] (7)  [below left of=5] {$13_{4}^2$};
  \node[captured_node] (8)  [right of=7] {$15_{3}^2$};

  

\end{tikzpicture}
\end{center}
\end{frame}


\begin{frame}{Il problema della sfida indiretta}
 4 sfida 71 e prima che 6 abbia risposto arriva anche la richiesta di sfida da parte di 3

\begin{center}
\begin{tikzpicture}[nodeStyle]

  \node[captured_node] (1)  {$43_{3}^3$};
  \node[passive_node] (2)  [above right of=1] {$5^2$};
  \node[captured_node] (3)  [below right of=2] {$71_{3}^3$};
  \node[passive_node] (4)  [above right of=3] {$6^2$};
  
  \node[candidate_node] (5)  [below of=1] {$4^4$};
  \node[candidate_node] (6)  [right of=5] {$3^4$};
  \node[captured_node] (7)  [below left of=5] {$13_{4}^2$};
  \node[captured_node] (8)  [right of=7] {$15_{3}^2$};

  

\end{tikzpicture}
\end{center} 
\end{frame}


\begin{frame}{Il problema della sfida indiretta}
4 sfida 5 e 3 sfida 6
\begin{center}
\begin{tikzpicture}[nodeStyle]



  \node[captured_node] (1)  {$43_{3}^3$};
  \node[candidate_node] (2)  [above right of=1] {$5_{4}^5$};
  \node[captured_node] (3)  [below right of=2] {$71_{3}^3$};
  \node[candidate_node] (4)  [above right of=3] {$6^4_{71}$};
  
  \node[leader_node] (5)  [below of=1] {$4^5$};
  \node[leader_node] (6)  [right of=5] {$3^5$};
  \node[captured_node] (7)  [below left of=5] {$13_{4}^2$};
  \node[captured_node] (8)  [right of=7] {$15_{3}^2$};

  

\end{tikzpicture}
\end{center} 
\end{frame}

\begin{frame}{Il problema della sfida indiretta}
 Questo problema può essere risolto inserendo in coda i messaggi che arrivano nei nodi catturati che hanno inoltrato un messaggio di sfida al proprio owner e che ancora non hanno ricevuto risposta.
 Questo garantisce che i 'territori' di due candidati non si sovrappongano.
 
\end{frame}