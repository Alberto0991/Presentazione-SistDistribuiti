\begin{frame}{Correttezza}

Il criterio di terminazione è corretto infatti:

\begin{itemize}
\setlength\itemsep{2em}
 \item I territori associati a due o più nodi \textit{CANDIDATE} non possono sovvrapporsi.
 \item Se un candiato raggiunge lo stage $\lfloor\frac{n}{2}\rfloor+1$ non esistente un'altro candidato nella rete con stage maggiore.
\end{itemize}

\end{frame}

\begin{frame}{Correttezza}
 Rimane da verificare che almeno un candidato raggiunga lo stage $\lfloor\frac{n}{2}\rfloor+1$. 

 Supponiamo per assurdo che nessun nodo raggiunga lo stage $\lfloor\frac{n}{2}\rfloor+1$.  Questo implica che tutti i candidati siano stati catturati (da altri candiati) o siano diventati passivi.
 Questo non è possibile perchè in ogni sfida, diretta o indiretta, viene sconfitto uno solo dei due candidati.
 
 
  
\end{frame}

