\begin{frame}{Correttezza}

Il criterio di terminazione è corretto infatti:

\begin{itemize}
\setlength\itemsep{2em}
 \item I territori associati a due o più nodi \textit{CANDIDATE} non possono sovvrapporsi.
 \item Se un candiato raggiunge lo stage $\lfloor\frac{n}{2}\rfloor+1$ non esiste un'altro candidato nella rete con stage maggiore.
\end{itemize}

\end{frame}

\begin{frame}{Correttezza}
 Rimane da verificare che almeno un candidato raggiunga lo stage $\lfloor\frac{n}{2}\rfloor+1$. 

 Supponiamo per assurdo che nessun nodo raggiunga lo stage $\lfloor\frac{n}{2}\rfloor+1$.  Questo implica che tutti i candidati siano stati catturati (da altri candidati) o siano diventati passivi.
 Questo non è possibile perchè:
 \begin{itemize}
  \item quando un candidato sfida un'altro candidato direttamente o indirettamente uno solo dei due 'muore'.
  \item se un candidato sfida un nodo passivo e il nodo passivo 'vince' allora esiste un'altro candidato con stage maggiore o uguale al suo. 
 \end{itemize}

 
 
  
\end{frame}

